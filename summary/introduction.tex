\section{Introduction}
How to uncover laws of nature?
\begin{itemize}
	\item practical $\rightarrow$ observation of experiments
	\item theoretical $\rightarrow$ relations ship between mathematical quantities
	\item[!] often experiments impossible (nuclear reactor, oil sill, too long, too short, where to measure) and mathematical equations too complex (only simplified models)
	\item[$\rightarrow$] numerical simulation: combines both approaches, see Figure~\ref{fig:simproc}
\end{itemize}

\subsection{Simulation Procedure}

\begin{figure}[H]
\begin{tikzpicture}
  [node distance=.8cm,
  start chain=going below,]
    \node[punktchain, join] (phys) {Physical Reality};
    \node[punktchain, join] (math)      {Mathematical Model};
    \node[punktchain, join] (discr)      {Discretization};
    \node[punktchain, join] (time-dept) {Time dependencies};
    \node[punktchain, join] (nonlin) {Nonlinearity};
  	\node[punktchain, join] (lin) {Linear system solvers};
    \node[punktchain, join] (error) {Error discrete solution};
    \node[punktchain, join] (eval) {Evaluation};
    \node[punktchain, join] (goal) {Goal: Gain insight};

    \node [punktchain, right=of phys] (obs) {Observation of Experiments};
	\node [punktchain, right=of math] (improvec) {Improvement of conitunous mathematical models};
    \node [punktchain, right=of discr] (improved) {Improvement of discretization};
    \node [punktchain, left=of time-dept, rotate=90] (parallel) {Parallelisation};
	
	\coordinate[right=of lin] (lin-right);
	\coordinate[right,above=of lin] (lin-top);
    
    \coordinate[right=of nonlin] (nonlin-right);

	\draw[->] (lin.east) |- (lin-top) -| (lin.north);
	\draw[->] (lin.east) -| (nonlin-right) -- (nonlin.east);
	\draw[->] (lin.east) -| (nonlin-right) |- (time-dept.east);

    \draw[->] (eval.east) -| (improved.south);
     
    \draw[->] (parallel.east) |- (discr.west);
    \draw[->] (parallel.south) |- (nonlin.west);
    \draw[->] (parallel.south) |- (time-dept.west);
    \draw[->] (parallel.south) |- (lin.west);
    \draw[->] (parallel.west) |- (error.west);

	\coordinate[right=of improvec] (improvec-coord);
	\draw[->] (eval.east) -| (improvec-coord) -- (improvec);

   	\draw[<->] (obs) -- (improvec);
    \draw[<->] (obs) -- (phys);   		   
 		   
  	\draw[->] (improvec) -- (math);   
   	\draw[->] (improved) -- (discr);
	

  \end{tikzpicture}
  
  \renewcommand{\thefigure}{1.1}
  \caption{Typical procedure in numerical simulation}
  \label{fig:simproc}
\end{figure}

  
\subsection{Fluids and Flows}

Fluids $\left\{
\begin{tabular}{c}
    gas \\
    liquids
\end{tabular}
\right\}$ are substances that cannot resist shear forces. \\
Examples: Water, Coffee, Air, Stream, Waterfall\\
Engineering: Flow around a car, flow pipeline. Often the iteration of a flow with a solid body, fluid sturcture interaction, multi-phase flows (gas $\leftrightarrow$ liquid).

\paragraph{Properties of Fluids}
\begin{itemize}
	\item Viscosity
	\begin{itemize}
		\item Frictional forces that act on the fluid
		\item[$\rightarrow$] Without external forces the fluid will come to a rest. The higher the viscosity, the faster it comes to a rest.
		\item Highly viscous (honey) $\rightarrow$ frictional forces are strong
		\item In gases viscous forces small $\rightarrow$ sometimes neglected
		\item[$\rightarrow$] "inviscid flow" in gas dynamics $\rightarrow$ Euler equation
	\end{itemize}
	\item Inertia
	\begin{itemize}
		\item[$\rightarrow$] Reynolds number : $\frac{\text{inertia}}{\text{viscosicty}}$
		\item[$\rightarrow$] high $\rightarrow$ turbulence; low $\rightarrow$ laminar
	\end{itemize}
	\item Compressibility
	\begin{itemize}
		\item Gases at high velocities are compressible
		\item Liquids are nearly incompressible
		\item[$\rightarrow$] Important if temperature changes
	\end{itemize}
\end{itemize}

Boundary layer theory
\begin{itemize}
	\item[$\rightarrow$] close to boundary, velocity is slow $\rightarrow$ viscous forces $\sim$ inertia forces
	\item[$\rightarrow$] away from boundaries, velocity high $\rightarrow$ viscous forces $<$ inertia forces
\end{itemize}

TODO pictures (2)

\subsection{Numerical Fluid Simulation}
Our goal: Simulate unsteady, incompressible, laminar flow, Navier-Stokes equations\\
\paragraph{History}
\begin{itemize}
	\item[$\rightarrow$] Dictated by compute power
	\item precursors: Crank-Nickelson '47
	\item real start: Harlow-Fromm '65
	\item in this course: "marker-and-cell" (MAC) method, Harlow-Welch '65
	\begin{itemize}
		\item simple
		\item flexible
		\item efficient
	\end{itemize}
	\item Alternatives: Finite elements, Finite Volumes, SIMPLE, QUICK
\end{itemize}
